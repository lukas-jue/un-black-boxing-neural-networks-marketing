\begin{figure}
\begin{center}
\def\layersep{3cm}

\begin{tikzpicture}[
   shorten >=1pt,->,
   draw=black!50,
    node distance=\layersep,
    every pin edge/.style={<-,shorten <=1pt},
    neuron/.style={circle,fill=black!25,minimum size=13pt,inner sep=0pt},
    input neuron/.style={neuron, fill=green!50},
    output neuron/.style={neuron, fill=red!50},
    hidden neuron/.style={neuron, fill=blue!50},
    annot/.style={text width=4em, text centered}
]

    % Draw the input layer nodes
    \foreach \name  [count=\y] in 
    {acad\_X1, age\_bin1, \dots,  age\_bin4, mobile\_logins, tenure, \dots, total\_mailings, transactions\_year}
     {   \node[input neuron, pin=left:\name] (I-\y) at (0,-\y*0.75 cm) {};  
     }

    % set number of hidden layers
    \newcommand\Nhidden{2}
    \newcommand\NodOne{5}
    \newcommand\NodTwo{10}
    \newcommand\Nod{9}

    % Draw the hidden layer nodes
%    \foreach \N in {1,...,\Nhidden} {

     \foreach \y in {1,...,\NodOne} {
          \path[yshift=-1.80cm]
              node[hidden neuron] (H1-\y) at (1*\layersep,-\y*0.75 cm) {};
              }
    \node[annot,above of=H1-1, node distance=2.8cm] (hl1) {Hidden layer 1};
     \foreach \y in {1,...,\NodTwo} {
          \path[yshift=0cm]
              node[hidden neuron] (H2-\y) at (2*\layersep,-\y*0.75 cm) {};            
           }
    \node[annot,above of=H2-1, node distance=1cm] (hl2) {Hidden layer 2};          

%    }

    \node[below=2.2cm of H1-\NodOne] (text1) {64 neurons};
    \node at (text1 -| H2-1) (text2) {128 neurons};
	\node at (text1 -| I-9) (text1) {74 inputs};

    % Draw the output layer node
    \node[output neuron,pin={[pin edge={->}]right:xsell}, right of=H\Nhidden-5] (O) {};

    \node at (text1 -| O) (text3) {1 output};
    % Connect every node in the input layer with every node in the
    % hidden layer.
    \foreach \source in {1,...,9}{
        \foreach \dest in {1,...,\NodOne}{
            \path (I-\source) edge (H1-\dest);
         }
    }
    % connect all hidden stuff
    \foreach [remember=\N as \lastN (initially 1)] \N in {2,...,\Nhidden}
       \foreach \source in {1,...,\NodOne}
           \foreach \dest in {1,...,\NodTwo}
               \path (H\lastN-\source) edge (H\N-\dest);

    % Connect every node in the hidden layer with the output layer
    \foreach \source in {1,...,\NodTwo}
        \path (H\Nhidden-\source) edge (O);

    % Annotate the layers

    \node[annot,left of=hl1] {Input layer};
    \node[annot,right of=hl\Nhidden] {Output layer};
\end{tikzpicture}
% End of code
\label{fig_nn_arch}
\caption{Architecture of the Artificial Neural Network implemented in this study}
\end{center}
\end{figure}